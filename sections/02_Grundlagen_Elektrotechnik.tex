\section{Grundlagen Elektrotechnik}

\subsection{Ladung}
Wichtigster Grundsatz: \textbf{Ladung ist eine Eigenschaft}.

Jedes Elementarteilchen hat eine Elementarladung. 
Daher ist jede gemessene Ladung immer ein Vielfaches dieser Elementarladung.

Die wichtigsten Eigenschaften sind:
\begin{itemize}
    \item Es gibt 2 Arten: Positiv \& Negativ
    \item Gleiche Ladungen stossen sich ab, ungleiche ziehen sich an
    \item Ladung ist quantisierbar (ein Vielfaches einer Elementarladung)
    \item Ladung bleibt insgesamt erhalten
    \item Ladung ist an Masse gebunden
\end{itemize}
Ladung wird in \unit{\coulomb}  $\widehat{=}$ \unit{\ampere\second} quantisiert.

\subsection{Ladungstransport}

Elektrische Ladung wird unterschiedlich gut in verschiedenen Medien transportiert.

\begin{center}
    \begin{tabular}{|l|l|l|} \hline  
        \textbf{Medium} & \textbf{Beweglicher Ladungsträger}    & \textbf{Beispiele} \\ \hline\hline 
            Metalle     & freie Elektronen                      & Cu, Ag, Au, ... \\\hline
            Elektrolyte & freie Ionen                           & Salzwasser, Säuren, Laugen, ... \\\hline
            Gase, Plasma& freie Elektronen, freie Ionen         & FL-Röhren, Sonnenplasma \\\hline
            Halbleiter  & Freie Elektronen, Löcher              & Si, Ge, Se, GaN, ... \\\hline
            Vakuum      & Keine(\say{nichts})                   & CRT-Bildschirme \\\hline
            Isolatoren  & \say{nichts}                          & Keramik, Kunststoffe, ... \\\hline
    \end{tabular}
\end{center}

Luft kann dabei zu den Gasen/Plasma und zu den Isolatoren gezählt werden, je nach Spannungsbereich

\subsection{Elektrischer Strom}

\begin{minipage}[c]{0.4\linewidth}
Elektrischer Strom ist der Transport von Ladung.
\[\text{Einfache Definition: I} = \dfrac{\Delta\text{Q}}{\Delta\textbf{t}} = \dfrac{\text{N}_\text{e}}{\Delta\textbf{t}}\]
\[\text{Masseinheit: }[\text{I}] = \dfrac{[\text{Q}]}{[\text{t}]} = \dfrac{\unit{\ampere\second}}{\unit{\second}} = \dfrac{\unit{\coulomb}}{\unit{\second}} = \unit{\ampere}\]

\end{minipage}
\begin{minipage}[c]{0.6\linewidth}
    Elektrischer Strom braucht immer eine Bezugsrichtung:
    \begin{center}
        \includesvg[width = 0.6\linewidth, inkscapelatex=false]{svg/Elektrischer Strom.svg}
    \end{center}

    Die Richtung des elektrischen Stromes sagt aus in welche Richtung sich die Elektronen bewegen. 

    Es wird zwischen der technischen und der physikalischen Richtung unterschieden, welche gegeneinander laufen.

    Die Elementarladung eines Elektrons ist negativ.
\end{minipage}

\subsection{Stromdichte (rechtwinklig zur Fläche)}

\begin{minipage}[t]{{0.5\linewidth}}
    \vspace*{0pt}
    \begin{center}
        \includesvg[width =0.9\linewidth, inkscapelatex=false]{svg/Stromdichte.svg}
    \end{center}
\end{minipage}%
\begin{minipage}[t]{{0.5\linewidth}}

    Stromdichte sagt aus wie viel Strom über einen Querschnitt fliesst. 

    In einem Leiter ohne konstanten Querschnitt bleibt die Stromdichte proportional zur Fläche konstant.

    \[\text{Masseinheit: }[\text{J}] = \dfrac{[\text{I}]}{[\text{A}]} = \dfrac{\unit{\ampere}}{\unit{\square\meter}}\]

\end{minipage}

\subsubsection{Schmelzstromdichte}

Die Schmelzstromdichte sagt aus, wie viel Strom durch einen Leiter fliessen muss, bis dieser anfängt zu schmelzen (typischerweise tausende \unit{\ampere}).

\subsection{Elektrisches Potenzial}

Elektrisches Potenzial ist Arbeit, die für den Spannungstransport zur Verfügung steht oder freigesetzt wird beim Transport der Ladung. 

\[ \text{Einfache Definition: } [\varphi] = \frac{\Delta[\text{W}]}{[\text{Q}]} = \frac{\text{J}}{\text{C}} = \unit{\volt} = \frac{\unit{\square\meter\kilogram}}{\unit{\cubic\second\ampere}}\]

\subsection{Elektrische Spannung}

Elektrische Spannung ist ein Potenzialunterschied. 
D.h. elektrische Spannung ist eine Differenz von Energie, die zum Ladungstransport zur Verfügung steht.

\[ \text{Einfache Definition: }[\text{U}] = \varphi_a - \varphi_b = \unit{\volt}   \]

\subsection{Kirchhoff current law / Kirchhoff I / KCL}

Die Summe aller Ströme in einem geschlossenen Knoten ist 0.

\begin{minipage}[t]{{0.25\linewidth}}
    \vspace*{0pt}
    \input{tikz/Ströme.tex}
\end{minipage}%
\begin{minipage}[t]{{0.75\linewidth}}
    \[ \sum_{n} \text{I}_n = 0\]
    Die einzige Bedingung ist, dass alle Ströme dieselbe Bezugsrichtung haben.
\end{minipage}

\subsection{Kirchhoff voltage law / Kirchhoff II / KVL}

Die Summe aller Teilspannungen in einer Masche ist 0. 

\[ \sum_{n} \text{U}_n = 0\]

Die Bedingung ist das alle Spannungen dasselbe Referenzpotential haben.

\subsection{Elektrische Leistung}

Elektrische Leistung ist Arbeit pro Zeit.

\[ \text{Definition :} [\text{P}] = \frac{\Delta[\text{W}]}{\Delta[\text{t}]} = \frac{\unit{\joule}}{\unit{\second}} = \frac{\unit{\volt\ampere\second}}{\unit{\second}} = \unit{\volt\ampere} = \unit{\watt} \]

\subsection{Verbraucher oder Quelle} \label{source_sink}

\begin{minipage}[t]{{0.5\linewidth}}
    \vspace*{0pt}
    \input{tikz/Verbraucher vs Quelle.tex}
\end{minipage}%
\begin{minipage}[t]{{0.5\linewidth}}
    Um einen Verbraucher und eine Quelle auseinander zu halten, kann man die Richtungen der Spannung und des Stromes überprüfen.

    Ist die Spannung und der Strom parallel zueinander, handelt es sich um einen Verbraucher (wie ein Widerstand).

    Ist die Spannung und der Strom antiparallel, handelt es sich um eine Quelle (aus einer Spannungsquelle fliesst ein Strom hinaus).
\end{minipage}

\subsection{Elektrischer Widerstand / Leitwert}

Elektrischer Widerstand / Leitwert ist eine Konsequenz des Zusammenhangs von elektrischem Strom und Spannung. 
\[\text{I}   \underbrace{\text{ist proportional zu}}_\propto{}   \text{U} \\ \text{I} = \text{G}\cdot\text{U}\]

G ist der sogenannte elektrische Leitwert und von oben abgeleitet: $\text{[G]} = \dfrac{\text{[I]}}{\text{[U]}} = \unit{\siemens} $ (Siemens).

Der elektrische Widerstand entsteht durch den Kehrwert vom Leitwert: $[\text{R}] = \dfrac{1}{[\text{G}]} = \dfrac{\text{[\text{U}]}}{[\text{I}]} = \unit{\ohm} $ (Ohm).

\subsection{Leitfähigkeit}

Elektrische Leitfähigkeit ist eine Materialeigenschaft und hängt von der Mobilität, der Anzahl und der Ladung der freien Ladungsträger ab.

\begin{align*}
    \text{Leitfähigkeit } [\sigma] & = \text{n}\cdot\text{q}\cdot\text{\textmu} = \unit[per-mode=fraction]{\ampere\per\volt\per\meter} = \unit[per-mode=fraction]{\siemens\per\meter}\\
    \text{n}& = \text{Anzahl freier Ladungsträger}\\
    \text{q}& = \text{Ladung pro Ladungsträger}\\
    \text{\textmu}& = \text{Mobilität / Freiheit der Elektronen sich zu bewegen}\\
\end{align*}

Die Faktoren \say{n} und \say{q} sind dabei Materialkonstanten. 
\say{\textmu} ist allerdings temperaturabhängig. 

\subsubsection{Mobilität}

Die Mobilität \textmu eines Ladungsträgers setzt sich wie folgt zusammen:
\begin{align*}
    \text{\textmu}& =  \frac{\text{q}\tau}{\text{m}}\\
    \text{q}& = \text{Ladung des Ladungsträger}\\
    \tau& = \text{Mittlere Stosszeit zwischen den Ladungsträger}\\
    \text{m}& = \text{Masse des Ladungsträger}\\
    \newline
    [\text{\textmu}]& = \unit[per-mode=fraction]{\coulomb\second\per\kilo\gram} = \unit[per-mode=fraction]{\square\meter\per\volt\per\second}
\end{align*}


\subsubsection{Spezifischer Widerstand}

Der Kehrwert der Leitfähigkeit wird oftmals auch verwendet:

\[ [\text{Spezifischer Widerstand}] = \left[ \frac{1}{\sigma} \right] = [\varrho] = \unit[per-mode=fraction]{\volt\meter\per\ampere}\]

\subsubsection{Widerstandsberechnungen}

Um aus Leitfähigkeit und spez. Widerstand einen Widerstand zu berechnen, können folgende Formeln verwendet werden.

\[
\unit{\ohm} = \frac{\varrho\cdot[\text{Leiterlänge}]}{[\text{Leiterfläche}]} = \frac{\varrho \cdot \unit{\meter}}{\unit{\square\meter}}\\
\unit{\ohm} = \frac{[\text{Leiterlänge}]}{\sigma\cdot[\text{Leiterfläche}]} = \frac{\unit{\meter}}{\sigma\cdot\unit{\square\meter}}\\
\]

Allerdings sind diese Formeln nicht allgemeingültig und können nur verwendet werden, wenn die Feldlinien konstant senkrecht zur Fläche stehen. 
Manchmal wird die Einheit auch gekürzt dargestellt (machen aber nur einige fiese Dozenten).

\subsection{Temperaturabhängigkeit}

Nicht ideale Widerstände ändern ihren Wert, unter anderem, abhängig mit der Temperatur. 
In der Regel ändert sich der Widerstand nicht linear und ist von Widerstand zu Widerstand etwas anders. 
Um trotzdem Widerstandswerte annähern zu können, wird die Funktion des Widerstands angenähert. 
Dies erfolgt je nach Anwendung mit einem, zwei oder noch mehr Termen.

\[
\begin{aligned}
    \Delta\text{R}& = \alpha\cdot\Delta\text{T}\cdot\text{R}_\text{t20} &&\rightarrow \text{Lineare Annäherung}\\
    \Delta\text{R}& = \alpha\cdot\Delta\text{T}\cdot\text{R}_\text{t20} + \beta\cdot(\Delta\text{T})^2\cdot\text{R}_\text{t20} &&\rightarrow \text{Quadratische Annäherung}\\
     &\text{ ...} & & \\
    \text{R}_{(\text{T})}& = (1 + \alpha\cdot\Delta\text{T})\cdot\text{R}_\text{t20} &&\rightarrow \text{Lineare Annäherung}\\
    \text{R}_{(\text{T})}& = (1 + \alpha\cdot\Delta\text{T} + \beta\cdot(\Delta\text{T})^2)\cdot\text{R}_\text{t20} &&\rightarrow \text{Quadratische Annäherung}
\end{aligned}
\]

Der Widerstand $\text{R}_\text{t20}$ setzt hier jeweils den \say{Nullpunkt} oder den Ausgangspunkt, von welcher aus die Temperatur berechnet wird.

\subsubsection{PPM's}

Eine andere, in der Praxis häufig verwendete Art, Temperaturkoeffizienten anzugeben, sind die sogenannten PPM's oder auch parts per milion (in der Praxis auch TCR / temperature coefficient). 
Diese sagen aus, um wie viel sich ein Widerstandswert pro Kelvin ändert. Daher handelt es sich hier nur um ein lineares Modell mit beschränkter Genauigkeit. 
Sie haben meist keine Potenz oder sind ganzzahlig, da sie, wie der Name es schon andeutet, mit $10^{-6}$ multipliziert werden. 
Die korrespondierende Formel dazu lautet:

\[
    \frac{\Delta\text{R}(\Delta\text{T})}{\text{R}_{\text{T0}}} = \frac{\Delta\text{R}}{10^{6}}
\]
