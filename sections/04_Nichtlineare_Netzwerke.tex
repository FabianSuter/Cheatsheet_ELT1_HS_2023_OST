\section{Nicht lineare Netzwerke / Leistungsanpassung}

\begin{minipage}[t]{{0.1\linewidth}}
    \vspace*{0pt}
    \begin{center}
        \begin{circuitikz}
            \draw (0,0) to[R] (0,1.5);
            \draw [short] (0.25,1.25) -- (0.25,1);
            \draw [short] (0.25,1) -- (-0.25,0.5);
            \draw [short] (-0.25,0.5) -- (-0.25,0.25);   
        \end{circuitikz}
    \end{center}
\end{minipage}%
\begin{minipage}[t]{{0.9\linewidth}}
    Viele Netzwerke sind in der Praxis nicht linear, daher wird meistens ein nicht linearer Widerstand verwendet, um diesen Effekt zu modellieren.

    Das links stehende Schemasymbol wird dabei häufig verwendet.

    Die Änderung des Widerstandswertes, welche durch einen Effekt wie zum Beispiel Temperatur, 
    Licht oder mechanische Einwirkung herbeigeführt werden, können mit Symbolen signalisiert werden, wie z.B. Pfeile oder ein Temperatursymbol.
\end{minipage}

\subsection{Leistungsanpassung}

Leistungsanpassung beschreibt das Verfahren, einen Widerstand an einer realen Quelle anzupassen, sodass die maximal mögliche Energie aus der Quelle fliesst. 
Im linearen Fall entspricht der Lastwiderstand dem Innenwiderstand der Quelle. 

\begin{minipage}[c]{0.49\linewidth}
    \input{circuits/Leistungsanpassung thevenin.tex}

    \begin{center}
        \begin{tcolorbox}[colframe=violet , colback=white, arc is curved, hbox]
            $R_l^\ast = R_i$
        \end{tcolorbox}
        maximale leistung, $\eta=50\%$\\
        Der Stern bei $R_l$ signalisiert, dass Leistungsanpassung gemacht wurde. 
    \end{center}
\end{minipage}
\hfill\vline\hfill
\begin{minipage}[c]{0.49\linewidth}
    \input{tikz/Leistungsanpassung.tex}
\end{minipage}

\subsubsection{Herleitung des optimalen Lastwiderstands einer linearen Stromquelle}

\begin{minipage}{0.3\linewidth}
    \begin{center}
        \input{circuits/leistungsanpassung norton.tex}
    \end{center}
\end{minipage}
\begin{minipage}{0.7\linewidth}
    \begin{align*}
        U_{R_l} & = \frac{R_i R_l }{R_i + R_l} \cdot I \\
        P_{R_l} & = \frac{1}{R_l}(U_{R_l})^2\\
        0 & = \frac{\diff P(R_l)}{\diff R_l} = R_l' (U_{R_l})^2 + 2\frac{1}{R_l}(U_{R_l'})\\
        0 & = -\frac{1}{R_l^2}\frac{R_i^2 R_l^2 I^2}{(R_i + R_l)^2 } + 2\frac{1}{R_l} \frac{R_i R_l I}{R_i + R_l}\left(\frac{R_i I(R_i + R_L)- R_i R_lI}{(R_i + R_l)^2}\right)\\
        0 & = 2 \frac{R_i^3 I^2}{(R_i + R_l)^3} - \frac{R_i^2  I^2}{(R_i + R_l)^2 }\\
        \frac{R_i^2  I^2}{(R_i + R_l)^2 } & = 2 \frac{R_i^3 I^2}{(R_i + R_l)^3}\\
        R_i^2 I^2 (R_i + R_l) & = 2 R_i^3 I^2\\
        2R_i & = R_i + R_l\\
        R_i &= R_l
    \end{align*}
\end{minipage}

Im Falle eines nicht linearen Widerstands muss eine Formel für die Leistung über dem Lastwiderstand gefunden werden. 
Die gefundene Formel muss dann abgeleitet und null gesetzt werden, ähnlich wie in der Herleitung.
